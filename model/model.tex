\documentclass[a4paper]{article}

\usepackage[utf8]{inputenc}
\usepackage[T1]{fontenc}
\usepackage[english]{babel}

\usepackage{amsmath}
\usepackage{amssymb}
\usepackage{amsfonts}
\usepackage{amsthm}
\usepackage{nicefrac}
\usepackage{enumerate}

\usepackage{graphicx}
\newcommand{\minus}{\scalebox{2}[1.0]{$-$}}

\title{FlightBBQ model}

\begin{document}

\maketitle

\section{Notations}

\paragraph{}
All along, we will make use of the following notations:
\begin{itemize}
\item $\mathsf{flighID}$ is a unique identifier to each flight
\item $\mathsf{delay}$ denotes by the time by which the flight was delayed. This variable may take negative values if the plane arrives before the announced arrival time. 
\item $\mathsf{delayed}$ denotes whether the flight was delayed
\item $\mathsf{departure}$ denotes the departure airport
\item $\mathsf{departureState}$ denotes the state of the departure airport
\item $\mathsf{arrival}$ denotes the arrival airport
\item $\mathsf{arrivalState}$ denotes the state of the arrival airport
\item $\mathsf{day}$ denotes the day the flight was operated, a.k.a. Monday, Tuesday, and so on
\item $\mathsf{month}$ denotes the month, a.k.a. Monday, Tuesday, and so on
\item $\mathsf{year}$ denotes the year, a.k.a. January, February, etc
\item $\mathsf{season}$ denotes the season, a.k.a. spring, summer, fall, winter
\item $\mathsf{carrier}$ denotes the airline
\item $\mathsf{tailID}$ denotes the tail number of the plane
\item $\mathsf{plane}$ denotes the plane model, a.k.a. Airbus 320, Boeing 737, and so on
\item $\mathsf{manufacturer}$  denotes the manufacturer of the plance, a.k.a. Airbus, Boeing, and so on
\item $\mathsf{distance}$ denotes the distance for the flight
\item $\mathsf{timeAir}$ denotes the time spent in the air
\item $\mathsf{cause}$ denotes whether it was a weather delay, a security delay, and so on.
\end{itemize}

\section{Model}

\paragraph{}
From a mathematical perspective, all the variable we consider will be treated as random variables, apart from $\mathsf{flighID}$ the flight identifier. We will mainly work with conditional expectation. For example, the expected delay for a flight operated by $\mathsf{carrier = United}$ on a Thursday is 
$$ \mathbb{E}[\mathsf{delay} \, | \, \mathsf{carrier = United}, \mathsf{day = Thursday}] $$

\paragraph{}
Conditional expectations $\mathbb{E}[U|V]$ benefit from a range of interesting properties. The first one is that $\mathbb{E}[U|V]$ is the best approximation of $U$ conditional to the knowledge of $V$ in the following sense. If $\mathcal{F}$ is the set of all measurable functions $f \; | \; \mathbb{R} \rightarrow \mathbb{R}$:
$$ \lVert U - \mathbb{E}[U|V] \rVert_{L_2} = \min_{f \in \mathcal{F}}  \lVert U - f(V) \rVert_{L_2}$$
The equality means that among the quantities $f(V)$ we can build from the data $V$, $\mathbb{E}[U|V]$ is the closest one to $U$.

\paragraph{}
From a computational perspective, $\mathbb{E}[U|V = v]$ is extremely easy to compute if we have enough data. Just select entries where $V = v$, then average $U$ over these entries. For example, if we want to compute the expected delay given a flight operated by $\mathsf{United}$, we select all the flight corresponding to $\mathsf{carrier = United}$ and we compute the average of the column $\mathsf{delay}$.

\paragraph{Warning}
This method of computation relies on the law of large numbers and therefore requires that we have enough data points when we compute the average. If we don't have enough data points or no point at all, we have to build a model that related $\mathsf{delay}$ to all other variables:
$$ \mathsf{delay} = f(\mathsf{carrier, day, \dots}) + \text{noise}$$
$f$ can implement a linear regression, a neural network, etc.

\end{document}